
\sloppy

\section{Initialization}

Routines beginning with the {\tt glutInit}- prefix are used to
initialize GLUT state.  The primary initialization routine is {\tt
glutInit} that should only be called exactly once in a GLUT program.
No non-{\tt glutInit}- prefixed GLUT or OpenGL routines should be
called before {\tt glutInit}.

The other {\tt glutInit}- routines may be called before {\tt
glutInit}.  The reason is these routines can be used to set default
window initialization state that might be modified by the command
processing done in {\tt glutInit}.  For example, {\tt
glutInitWindowSize(400, 400)}\index{glutInitWindowSize} can be called
before {\tt glutInit} to indicate 400 by 400 is the program's default
window size.  Setting the {\em initial window size} or {\em position} before {\tt glutInit}
allows the GLUT program user to specify the initial size or position using
command line arguments.

\subsection{glutInit}

\rndx{glutInit} is used to initialize the GLUT library.

\subsubsection*{Usage}

\begin{verbatim}
void glutInit(int *argcp, char **argv);
\end{verbatim}

\begin{description}
\itemsep 0in
\item[{\tt argcp}] A pointer to the program's
{\em unmodified} {\tt argc} variable from {\tt main.}
Upon return, the value pointed to by {\tt argcp}
will be updated, because {\tt glutInit}
extracts any command line options intended for the GLUT library.

\item[{\tt argv}] The program's {\em unmodified} {\tt argv} variable from
{\tt main.} Like {\tt argcp,} the data for {\tt argv}
will be updated because {\tt glutInit}
extracts any command line options understood by the GLUT library.
\end{description}

\subsubsection*{Description}

{\tt glutInit} will initialize the GLUT library and negotiate a session with the window
system.  During this process, {\tt glutInit}
may cause the termination of the GLUT program with an error message to the user if GLUT
cannot be properly initialized.  Examples of this situation include the failure to
connect to the window system, the lack of window system support for OpenGL, and invalid
command line options.

{\tt glutInit} also processes command line options, but the specific options
parse are window system dependent.

\subsubsection*{X Implementation Notes}

The X Window System specific options parsed by {\tt glutInit} are as follows:
\begin{description}
\item[{\tt -display} {\em DISPLAY}]
Specify the X server to connect to.  If not specified, the
value of the {\tt DISPLAY} environment variable is used.

\item[{{\tt -geometry} {\em W}{\tt x}{\em H}{\tt +}{\em X}{\tt +}{\em Y}}]
Determines where window's should be created on the screen.  The
parameter following {\tt -geometry} should be formatted as a standard X geometry specification.
The effect of using this option is to change the GLUT
{\em initial size} and {\em initial position} the same as if
{\tt glutInitWindowSize} or {\tt glutInitWindowPosition}
were called directly.

\item[{\tt -iconic}]
Requests all top-level windows be created in an iconic state.

\item[{\tt -indirect}]
Force the use of {\em indirect} OpenGL rendering contexts.

\item[{\tt -direct}]
Force the use of {\em direct}
OpenGL rendering contexts (not all GLX implementations support
direct rendering contexts).  A fatal error is generated if direct
rendering is not supported by the OpenGL implementation.

If neither {\tt -indirect} or {\tt -direct} are used to force
a particular behavior, GLUT will attempt to use direct rendering
if possible and otherwise fallback to indirect rendering.

\item[{\tt -gldebug}]
After processing callbacks and/or events, check if there are any OpenGL
errors by calling {\tt glGetError}.  If an error is reported, print out
a warning by looking up the error code with {\tt gluErrorString}.  Using
this option is helpful in detecting \index{OpenGL errors} OpenGL run-time errors.

\item[{\tt -sync}]
Enable synchronous X protocol transactions.  This option makes it easier
to track down potential \index{X protocol errors} X protocol errors.

\end{description}

\subsection{glutInitWindowPosition, glutInitWindowSize}

\rndx{glutInitWindowPosition} and \rndx{glutInitWindowSize} set the
{\em initial window position} and {\em size} respectively.

\subsubsection*{Usage}

\begin{verbatim}
void glutInitWindowSize(int width, int height);
void glutInitWindowPosition(int x, int y);
\end{verbatim}
\begin{description}
\itemsep 0in
\item[{\tt width}]
Width in pixels.
\item[{\tt height}]
Height in pixels.
\item[{\tt x}]
Window X location in pixels.
\item[{\tt y}]
Window Y location in pixels.
\end{description}

\subsubsection*{Description}

Windows created by {\tt glutCreateWindow} will be requested to be created
with the current {\em initial window position} and {\em size}.

The initial value of the {\em initial window position} GLUT state is
-1 and -1.  If either the X or Y component to the {\em initial
window position} is negative, the actual window position is left
to the window system to determine.
The initial value of the {\em initial window size} GLUT state is 300 by 300.
The {\em initial window size} components must be greater than zero.

The intent of the {\em initial window position} and {\em size} values is
to provide a suggestion to the window system for a window's initial size
and position.  The window system is not obligated to use this information.
Therefore, GLUT programs should not assume the window was created at the
specified size or position.  A GLUT program should use the window's
reshape callback to determine the true size of the window.

\subsection{glutInitDisplayMode}

\rndx{glutInitDisplayMode} sets the {\em initial display mode}.

\subsubsection*{Usage}

\begin{verbatim}
void glutInitDisplayMode(unsigned int mode);
\end{verbatim}
\begin{description}
\itemsep 0in
\item[{\tt mode}]
Display mode, normally the bitwise {\em OR}-ing of GLUT display mode bit masks.
See values below:
\end{description}
\begin{description}
\itemsep 0in
\item[{\tt GLUT\_RGBA}]
Bit mask to select an RGBA mode window.  This is the default if neither {\tt GLUT\_RGBA}
nor {\tt GLUT\_INDEX} are specified.
\item[{\tt GLUT\_RGB}]
An alias for {\tt GLUT\_RGBA}.
\item[{\tt GLUT\_INDEX}]
Bit mask to select a color index mode window.  This overrides {\tt GLUT\_RGBA} if it is also
specified.
\item[{\tt GLUT\_SINGLE}]
Bit mask to select a single buffered window.  This is the default if neither {\tt GLUT\_DOUBLE}
or {\tt GLUT\_SINGLE} are specified.
\item[{\tt GLUT\_DOUBLE}]
Bit mask to select a double buffered window.  This overrides {\tt GLUT\_SINGLE} if it is also
specified.
\item[{\tt GLUT\_ACCUM}]
Bit mask to select a window with an accumulation buffer.
\item[{\tt GLUT\_ALPHA}]
Bit mask to select a window with an alpha component to the color buffer(s).
\item[{\tt GLUT\_DEPTH}]
Bit mask to select a window with a depth buffer.
\item[{\tt GLUT\_STENCIL}]
Bit mask to select a window with a stencil buffer.
\item[{\tt GLUT\_MULTISAMPLE}]
Bit mask to select a window with multisampling support.  If multisampling is
not available, a non-multisampling window will automatically be chosen.
Note:  both the OpenGL client-side and server-side implementations must support
the {\tt GLX\_SAMPLE\_SGIS} extension for multisampling to be available.
\item[{\tt GLUT\_STEREO}]
Bit mask to select a stereo window.
\item[{\tt GLUT\_LUMINANCE}]
Bit mask to select a window with a ``luminance'' color model.  This
model provides the functionality of OpenGL's RGBA color model, but the
green and blue components are not maintained in the frame buffer.
Instead each pixel's red component is converted to an index between zero and
{\tt glutGet(GLUT\_WINDOW\_COLORMAP\_SIZE)-1} and looked up in a per-window
color map to determine the color of pixels within the window.
The initial colormap of {\tt GLUT\_LUMINANCE} windows is
initialized to be a linear gray ramp, but can be modified with
GLUT's colormap routines.
\end{description}

\subsubsection*{Description}

The {\em initial display mode} is used when creating top-level windows,
subwindows, and overlays to determine the OpenGL display mode for the to-be-created window
or overlay.

Note that {\tt GLUT\_RGBA} selects the RGBA color model, but it does
not request any bits of alpha (sometimes called an {\em alpha buffer}
or {\em destination alpha}) be allocated.  To request alpha, specify {\tt
GLUT\_ALPHA}.  The same applies to {\tt GLUT\_LUMINANCE}.

\subsubsection*{{\tt GLUT\_LUMINANCE} Implementation Notes}

\rndx{GLUT\_LUMINANCE} is not supported on most OpenGL platforms.

\section{Beginning Event Processing}

After a GLUT program has done initial setup such as creating
windows and menus, GLUT programs enter the GLUT event processing loop
by calling {\tt glutMainLoop}.

\subsection{glutMainLoop}

\rndx{glutMainLoop} enters the GLUT event processing loop.

\subsubsection*{Usage}

\begin{verbatim}
void glutMainLoop(void);
\end{verbatim}
 
\subsubsection*{Description}

{\tt glutMainLoop} enters the GLUT event processing loop.
This routine should be called at most once in a GLUT program.  Once called, this routine
will never return.  It will call as necessary any callbacks that have been registered.

\section{Window Management}

GLUT supports two types of windows:  top-level windows and subwindows.
Both types support OpenGL rendering and GLUT callbacks.  There is a
single identifier space for both types of windows.

\subsection{glutCreateWindow}

\rndx{glutCreateWindow} creates a top-level window.

\subsubsection*{Usage}
\begin{verbatim}
int glutCreateWindow(const char *name);
\end{verbatim}
\begin{description}
\itemsep 0in
\item[{\tt name}]
ASCII character string for use as window name.
\end{description}

\subsubsection*{Description}
{\tt glutCreateWindow} creates a top-level window.  The {\tt name} will
be provided to the window system as the window's name.  The intent is that the
window system will label the window with the name.

Implicitly, the {\em current window} is set to the newly created window.

Each created window has a unique associated OpenGL context.  State changes to
a window's associated OpenGL context can be done immediately after the
window is created.

The {\em display state} of a window is initially for the window
to be shown.  But the window's {\em display state} is not actually
acted upon until {\tt glutMainLoop} is entered.  This means until {\tt glutMainLoop}
is called, rendering to a created window is ineffective because the window
can not yet be displayed.

The value returned is a unique small integer identifier for the window.  The
range of allocated identifiers starts at one.  This window identifier
can be used when calling {\tt glutSetWindow}.

\subsubsection*{X Implementation Notes}

The proper X Inter-Client Communication Conventions Manual
(ICCCM)\index{X Inter-Client Communication Conventions Manual}
top-level properties are established.
The \rndx{WM\_COMMAND} property that lists the command line used to
invoke the GLUT program is only established for the first window created.

\subsection{glutCreateSubWindow}

\rndx{glutCreateSubWindow} creates a subwindow.

\subsubsection*{Usage}
\begin{verbatim}
int glutCreateSubWindow(int win,
                        int x, int y, int width, int height);
\end{verbatim}
\begin{description}
\itemsep 0in
\item[{\tt win}]
Identifier of the subwindow's parent window.
\item[{\tt x}]
Window X location in pixels relative to parent window's origin.
\item[{\tt y}]
Window Y location in pixels relative to parent window's origin.
\item[{\tt width}]
Width in pixels.
\item[{\tt height}]
Height in pixels.
\end{description}

\subsubsection*{Description}

{\tt glutCreateSubWindow} creates a subwindow of the window identified by {\tt win}
of size {\tt width} and {\tt height} at location {\tt x} and {\tt y}
within the {\em current window}.
Implicitly, the {\em current window} is set to the newly created subwindow.

Each created window has a unique associated OpenGL context.  State changes to
a window's associated OpenGL context can be done immediately after the
window is created.

The {\em display state} of a window is initially for the window
to be shown.  But the window's {\em display state} is not actually
acted upon until {\tt glutMainLoop} is entered.  This means until {\tt glutMainLoop}
is called, rendering to a created window is ineffective.
Subwindows can not be iconified.

Subwindows can be nested arbitrarily deep.

The value returned is a unique small integer identifier for the window.  The
range of allocated identifiers starts at one.

\subsection{glutSetWindow, glutGetWindow}

\rndx{glutSetWindow} sets the {\em current window}; \rndx{glutGetWindow}
returns the identifier of the {\em current window}.

\subsubsection*{Usage}
\begin{verbatim}
void glutSetWindow(int win);
int glutGetWindow(void);
\end{verbatim}
\begin{description}
\itemsep 0in
\item[{\tt win}]
Identifier of GLUT window to make the {\em current window}.
\end{description}

\subsubsection*{Description}

{\tt glutSetWindow} sets the {\em current window}; {\tt glutGetWindow}
returns the identifier of the {\em current window}.  If no windows
exist or the previously {\em current window} was destroyed, {\tt
glutGetWindow} returns zero.  {\tt glutSetWindow} does {\em not}
change the {\em layer in use} for the window; this is done using {\tt
glutUseLayer}.

\subsection{glutDestroyWindow}

\rndx{glutDestroyWindow} destroys the specified window.

\subsubsection*{Usage}
\begin{verbatim}
void glutDestroyWindow(int win);
\end{verbatim}
\begin{description}
\itemsep 0in
\item[{\tt win}]
Identifier of GLUT window to destroy.
\end{description}

\subsubsection*{Description}

{\tt glutDestroyWindow} destroys the window specified by {\tt win} and the window's
associated OpenGL context, logical colormap (if the window is color index), and overlay
and related state (if an overlay has been established).  Any subwindows of destroyed
windows are also destroyed by {\tt glutDestroyWindow}.
If {\tt win} was
the {\em current window}, the {\em current window} becomes
invalid ({\tt glutGetWindow} will return zero).

\subsection{glutPostRedisplay}

\rndx{glutPostRedisplay} marks the {\em current window} as needing to be redisplayed.

\subsubsection*{Usage}
\begin{verbatim}
void glutPostRedisplay(void);
\end{verbatim}

\subsubsection*{Description}

Mark the normal plane of {\em current window} as needing to be redisplayed.  The next iteration
through {\tt glutMainLoop}, the window's display callback will be called
to redisplay the window's normal plane.  Multiple calls to {\tt glutPostRedisplay}
before the next display callback opportunity generates only a single
redisplay callback.  {\tt glutPostRedisplay} may be called within a window's
display or overlay display callback to re-mark that window for redisplay.

Logically, normal plane damage notification for a window is treated as a
{\tt glutPostRedisplay} on the damaged window.  Unlike damage reported by
the window system, {\tt glutPostRedisplay} will {\em not} set to true the normal
plane's damaged status (returned by {\tt glutLayerGet(GLUT\_NORMAL\_DAMAGED)}.

Also, see {\tt glutPostOverlayRedisplay}.

\subsection{glutSwapBuffers}

\rndx{glutSwapBuffers} swaps the buffers of the {\em current window} if double buffered.

\subsubsection*{Usage}
\begin{verbatim}
void glutSwapBuffers(void);
\end{verbatim}

\subsubsection*{Description}

Performs a buffer swap on the {\em layer in use} for the {\em current window}.  Specifically, {\tt glutSwapBuffers}
promotes the contents of the back buffer of the {\em layer in use} of the {\em current window} to become the
contents of the front buffer.  The contents of the back buffer then
become undefined.  The update typically takes place during the vertical
retrace of the monitor, rather than immediately after {\tt glutSwapBuffers}
is called.

An implicit \rndx{glFlush} is done by {\tt glutSwapBuffers} before it returns.
Subsequent OpenGL commands can be issued immediately after calling
{\tt glutSwapBuffers}, but are not executed until the buffer exchange
is completed.

If the {\em layer in use} is not double buffered, {\tt glutSwapBuffers} has no effect.

\subsection{glutPositionWindow}

\rndx{glutPositionWindow} requests a change to the position of
the {\em current window}.

\subsubsection*{Usage}
\begin{verbatim}
void glutPositionWindow(int x, int y);
\end{verbatim}
\begin{description}
\itemsep 0in
\item[{\tt x}]
New X location of window in pixels.
\item[{\tt y}]
New Y location of window in pixels.
\end{description}

\subsubsection*{Description}

{\tt glutPositionWindow} requests a change in the position of
the {\em current window}.
For top-level windows, the {\tt x} and {\tt y} parameters are
pixel offsets from the screen origin.  For subwindows, the
{\tt x} and {\tt y} parameters are pixel offsets from the window's
parent window origin.

The requests by {\tt glutPositionWindow} are not processed immediately.
The request is executed after returning to the main event loop.
This allows multiple {\tt glutPositionWindow}, {\tt glutReshapeWindow},
and {\tt glutFullScreen} requests to the same window to be coalesced.

In the case of top-level windows, a {\tt glutPositionWindow} call
is considered only a request for positioning the window.
The window system is free to apply its own policies to top-level
window placement.  The intent is that top-level windows
should be repositioned according {\tt glutPositionWindow}'s parameters.

{\tt glutPositionWindow} disables the full screen status of a window
if previously enabled.

\subsection{glutReshapeWindow}

\rndx{glutReshapeWindow} requests a change to the size of
the {\em current window}.

\subsubsection*{Usage}
\begin{verbatim}
void glutReshapeWindow(int width, int height);
\end{verbatim}
\begin{description}
\itemsep 0in
\item[{\tt width}]
New width of window in pixels.
\item[{\tt height}]
New height of window in pixels.
\end{description}

\subsubsection*{Description}

{\tt glutReshapeWindow} requests a change in the size of
the {\em current window}.
The {\tt width} and {\tt height} parameters are
size extents in pixels.  The {\tt width} and {\tt height} must
be positive values.

The requests by {\tt glutReshapeWindow} are not processed immediately.
The request is executed after returning to the main event loop.
This allows multiple {\tt glutReshapeWindow}, {\tt glutPositionWindow},
and {\tt glutFullScreen}
requests to the same window to be coalesced.

In the case of top-level windows, a {\tt glutReshapeWindow} call
is considered only a request for sizing the window.
The window system is free to apply its own policies to top-level
window sizing.  The intent is that top-level windows
should be reshaped according {\tt glutReshapeWindow}'s parameters.
Whether a reshape actually takes effect and, if so, the reshaped dimensions
are reported to the program by a reshape callback.

{\tt glutReshapeWindow} disables the full screen status of a window
if previously enabled.

\subsection{glutFullScreen}

\rndx{glutFullScreen} requests that the {\em current window} be made full screen.

\subsubsection*{Usage}
\begin{verbatim}
void glutFullScreen(void);
\end{verbatim}

\subsubsection*{Description}

{\tt glutFullScreen} requests that the {\em current window} be made
full screen.  The exact semantics of what full screen means may vary by
window system.  The intent is to make the window as large as possible
and disable any window decorations or borders added the window
system.  The window width and height are not guaranteed to be the same
as the screen width and height, but that is the intent of making a
window full screen.

{\tt glutFullScreen} is defined to work only on top-level windows.

The {\tt glutFullScreen} requests are not processed immediately.
The request is executed after returning to the main event loop.
This allows multiple {\tt glutReshapeWindow}, {\tt glutPositionWindow},
and {\tt glutFullScreen} requests to the same window to be coalesced.

Subsequent {\tt glutReshapeWindow} and {\tt glutPositionWindow} requests
on the window will disable the full screen status of the window.

\subsubsection*{X Implementation Notes}

In the X implementation of GLUT, full screen is implemented by sizing
and positioning the window to cover the entire screen and posting the
\rndx{\_MOTIF\_WM\_HINTS} property on the window requesting absolutely no
decorations.  Non-Motif window managers may not respond to {\tt
\_MOTIF\_WM\_HINTS}.

\subsection{glutPopWindow, glutPushWindow}

\rndx{glutPopWindow} and \rndx{glutPushWindow} change the stacking order
of the {\em current window} relative to its siblings.

\subsubsection*{Usage}
\begin{verbatim}
void glutPopWindow(void);
void glutPushWindow(void);
\end{verbatim}

\subsubsection*{Description}

{\tt glutPopWindow} and {\tt glutPushWindow} work on both top-level
windows and subwindows.  The effect of pushing and popping windows
does not take place immediately.  Instead the push or pop is saved
for execution upon return to the GLUT event loop.  Subsequent
push or pop requests on a window replace the previously saved
request for that window.  The effect of pushing and popping
top-level windows is subject to the window system's policy for
restacking windows.

\subsection{glutShowWindow, glutHideWindow, glutIconifyWindow}

\rndx{glutShowWindow}, \rndx{glutHideWindow}, and \rndx{glutIconifyWindow}
change the display status of the {\em current window}.

\subsubsection*{Usage}
\begin{verbatim}
void glutShowWindow(void);
void glutHideWindow(void);
void glutIconifyWindow(void);
\end{verbatim}

\subsubsection*{Description}

{\tt glutShowWindow} will show the {\em current window} (though
it may still not be visible if obscured by other shown windows).  {\tt glutHideWindow} will hide
the {\em current window}.  {\tt glutIconifyWindow} will iconify
a top-level window, but GLUT prohibits iconification of a subwindow.
The effect of showing, hiding, and iconifying windows does not
take place immediately.  Instead the requests are saved
for execution upon return to the GLUT event loop.
Subsequent show, hide, or iconification requests on a window
replace the previously saved request for that window.  The effect of hiding,
showing, or iconifying top-level windows is subject to the window
system's policy for displaying windows.

\subsection{glutSetWindowTitle, glutSetIconTitle}

\rndx{glutSetWindowTitle} and \rndx{glutSetIconTitle} change
the window or icon title respectively of the current top-level window.

\subsubsection*{Usage}
\begin{verbatim}
void glutSetWindowTitle(const char *name);
void glutSetIconTitle(const char *name);
\end{verbatim}
\begin{description}
\itemsep 0in
\item[{\tt name}]
ASCII character string for the window or icon name to be set for the window.
\end{description}

\subsubsection*{Description}

These routines should be called only when the {\em current window}
is a top-level window.  Upon creation of a top-level window,
the window and icon names are determined by the {\tt name}
parameter to {\tt glutCreateWindow}.  Once created, {\tt glutSetWindowTitle}
and {\tt glutSetIconTitle} can change the window and icon names
respectively of top-level windows.  Each call requests the
window system change the title appropriately.  Requests are not
buffered or coalesced.  The policy by which the window and icon name
are displayed is window system dependent.

\subsection{glutSetCursor}

\rndx{glutSetCursor} changes the cursor image of the {\em current window}.

\subsubsection*{Usage}
\begin{verbatim}
void glutSetCursor(int cursor);
\end{verbatim}
\begin{description}
\itemsep 0in
\item[{\tt cursor}]
Name of cursor image to change to.
\end{description}

\begin{description}
\itemsep 0in
\item[{\tt GLUT\_CURSOR\_RIGHT\_ARROW}]
Arrow pointing up and to the right. 
\item[{\tt GLUT\_CURSOR\_LEFT\_ARROW}]
Arrow pointing up and to the left.
\item[{\tt GLUT\_CURSOR\_INFO}]
Pointing hand.
\item[{\tt GLUT\_CURSOR\_DESTROY}]
Skull \& cross bones.
\item[{\tt GLUT\_CURSOR\_HELP}]
Question mark.
\item[{\tt GLUT\_CURSOR\_CYCLE}]
Arrows rotating in a circle.
\item[{\tt GLUT\_CURSOR\_SPRAY}]
Spray can.
\item[{\tt GLUT\_CURSOR\_WAIT}]
Wrist watch.
\item[{\tt GLUT\_CURSOR\_TEXT}]
Insertion point cursor for text.
\item[{\tt GLUT\_CURSOR\_CROSSHAIR}]
Simple cross-hair.
\item[{\tt GLUT\_CURSOR\_UP\_DOWN}]
Bi-directional pointing up \& down.
\item[{\tt GLUT\_CURSOR\_LEFT\_RIGHT}]
Bi-directional pointing left \& right.
\item[{\tt GLUT\_CURSOR\_TOP\_SIDE}]
Arrow pointing to top side.
\item[{\tt GLUT\_CURSOR\_BOTTOM\_SIDE}]
Arrow pointing to bottom side.
\item[{\tt GLUT\_CURSOR\_LEFT\_SIDE}]
Arrow pointing to left side.
\item[{\tt GLUT\_CURSOR\_RIGHT\_SIDE}]
Arrow pointing to right side.
\item[{\tt GLUT\_CURSOR\_TOP\_LEFT\_CORNER}]
Arrow pointing to top-left corner. 
\item[{\tt GLUT\_CURSOR\_TOP\_RIGHT\_CORNER}]
Arrow pointing to top-right corner.
\item[{\tt GLUT\_CURSOR\_BOTTOM\_RIGHT\_CORNER}]
Arrow pointing to bottom-left corner.
\item[{\tt GLUT\_CURSOR\_BOTTOM\_LEFT\_CORNER}]
Arrow pointing to bottom-right corner.
\item[{\tt GLUT\_CURSOR\_FULL\_CROSSHAIR}]
Full-screen cross-hair cursor (if possible, otherwise {\tt GLUT\_CURSOR\_CROSSHAIR}).
\item[{\tt GLUT\_CURSOR\_NONE}]
Invisible cursor.
\item[{\tt GLUT\_CURSOR\_INHERIT}]
Use parent's cursor.
\end{description}

\subsubsection*{Description}

\rndx{glutSetCursor} changes the cursor image of the {\em current window}.
Each call requests the window system change the cursor appropriately.
The cursor image when a window is created is {\tt GLUT\_CURSOR\_INHERIT}.
The exact cursor images used are implementation dependent.  The intent is
for the image to convey the meaning of the cursor name.
For a top-level window, {\tt GLUT\_CURSOR\_INHERIT} uses the default window
system cursor.

\subsubsection*{X Implementation Notes}

GLUT for X uses SGI's \rndx{\_SGI\_CROSSHAIR\_CURSOR} convention
\cite{kilgard93-xsgi} to access a full screen cross-hair cursor if possible.

\section{Overlay Management}

When \ndx{overlay hardware} is available, GLUT provides a set of routine for
establishing, using, and removing an overlay for GLUT windows.
When an overlay is established, a separate OpenGL context is also
established.  A window's overlay OpenGL state is kept 
distinct from the normal planes OpenGL state.

\subsection{glutEstablishOverlay}

\rndx{glutEstablishOverlay} establishes an overlay (if possible) for
the {\em current window}.

\subsubsection*{Usage}

\begin{verbatim}
void glutEstablishOverlay(void);
\end{verbatim}

\subsubsection*{Description}

{\tt glutEstablishOverlay} establishes an overlay (if possible) for
the {\em current window}.  The requested display mode for the overlay is 
determined by the {\em initial display mode}.
{\tt glutLayerGet(GLUT\_OVERLAY\_POSSIBLE)} can be called to determine
if an overlay is possible for the {\em current window} with the
current {\em initial display mode}.  Do not attempt to establish an
overlay when one is not possible; GLUT will terminate the program.

If {\tt glutEstablishOverlay}
is called when an overlay already exists, the existing overlay is first
removed, and then a new overlay is established.  The state of the old overlay's
OpenGL context is discarded.

The initial display state of an overlay is shown, however the overlay
is only actually shown if the overlay's window is shown.

Implicitly, the window's {\em layer in use} changes to the overlay immediately
after the overlay is established.

\subsubsection*{X Implementation Notes}

GLUT for X uses the \rndx{SERVER\_OVERLAY\_VISUALS} convention \cite{kilgard93-layers}
is used to determine if overlay visuals are available.  While the convention
allows for opaque overlays (no transparency) and overlays with the transparency
specified as a bitmask, GLUT overlay management only provides access to transparent
pixel overlays.

Until RGBA overlays are better understood, GLUT only supports color index overlays.

\subsection{glutUseLayer}

\rndx{glutUseLayer} changes the {\em layer in use} for the {\em current window}.

\subsubsection*{Usage}

\begin{verbatim}
void glutUseLayer(GLenum layer);
\end{verbatim}
\begin{description}
\itemsep 0in
\item[{\tt layer}]
Either {\tt GLUT\_NORMAL} or {\tt GLUT\_OVERLAY}, selecting the normal plane or
overlay respectively.
\end{description}

\subsubsection*{Description}

{\tt glutUseLayer} changes the per-window {\em layer in use} for the
{\em current window}, selecting either the normal plane or overlay.
The overlay should only be specified if an overlay exists, however
windows without an overlay may still call {\tt
glutUseLayer(GLUT\_NORMAL)}.  OpenGL commands for the window are
directed to the current {\em layer in use}.

To query the {\em layer in use} for a window, call {\tt glutLayerGet(GLUT\_LAYER\_IN\_USE)}.

\subsection{glutRemoveOverlay}

\rndx{glutRemoveOverlay} removes the overlay (if one exists) from
the {\em current window}.

\subsubsection*{Usage}

\begin{verbatim}
void glutRemoveOverlay(void);
\end{verbatim}

\subsubsection*{Description}

{\tt glutRemoveOverlay} removes the overlay (if one exists).  It is
safe to call {\tt glutRemoveOverlay} even if no overlay is currently
established--it does nothing in this case.  Implicitly, the window's
{\em layer in use} changes to the normal plane immediately once the
overlay is removed.

If the program intends to re-establish the overlay later, it is
typically faster and less resource intensive to use {\tt glutHideOverlay}
and {\tt glutShowOverlay} to simply change the display status of the
overlay.

\subsection{glutPostOverlayRedisplay}

\rndx{glutPostOverlayRedisplay} marks the overlay of the {\em current
window} as needing to be redisplayed.

\subsubsection*{Usage}

\begin{verbatim}
void glutPostOverlayRedisplay(void);
\end{verbatim}

\subsubsection*{Description}

Mark the overlay of {\em current window} as needing to be redisplayed.  The next iteration
through {\tt glutMainLoop}, the window's overlay display callback (or simply the
display callback if no overlay display callback is registered) will be called
to redisplay the window's overlay plane.  Multiple calls to {\tt glutPostOverlayRedisplay}
before the next display callback opportunity (or overlay display callback opportunity
if one is registered) generate only a single
redisplay.  {\tt glutPostOverlayRedisplay} may be called within a window's
display or overlay display callback to re-mark that window for redisplay.

Logically, overlay damage notification for a window is treated as a
{\tt glutPostOverlayRedisplay} on the damaged window.  Unlike damage reported by
the window system, {\tt glutPostOverlayRedisplay} will not set to true the overlay's
damaged status (returned by {\tt glutLayerGet(GLUT\_OVERLAY\_DAMAGED)}.

Also, see {\tt glutPostRedisplay}.

\subsection{glutShowOverlay, glutHideOverlay}

\rndx{glutShowOverlay} shows the overlay of the {\em current window};
\rndx{glutHideOverlay} hides the overlay.

\subsubsection*{Usage}

\begin{verbatim}
void glutShowOverlay(void);
void glutHideOverlay(void);
\end{verbatim}

\subsubsection*{Description}

\rndx{glutShowOverlay} shows the overlay of the {\em current window};
\rndx{glutHideOverlay} hides the overlay.  The effect of showing or
hiding an overlay takes place immediately.  Note that {\tt
glutShowOverlay} will not actually display the overlay unless the
window is also shown (and even a shown window may be obscured by other
windows, thereby obscuring the overlay).  It is typically faster and
less resource intensive to use these routines to control the display
status of an overlay as opposed to removing and re-establishing the
overlay.

\section{Menu Management}

GLUT supports simple cascading pop-up menus.  They are designed to let a user
select various modes within a program.  The functionality
is simple and minimalistic and is meant to be that way.  Do not mistake GLUT's pop-up
menu facility with an attempt to create a full-featured user interface.

It is illegal to create or destroy menus, or change, add, or remove menu
items while a menu (and any cascaded sub-menus) are in use (that is,
popped up).

\subsection{glutCreateMenu}

\rndx{glutCreateMenu} creates a new pop-up menu.

\subsubsection*{Usage}

\begin{verbatim}
int glutCreateMenu(void (*func)(int value));
\end{verbatim}
\begin{description}
\itemsep 0in
\item[{\tt func}]
The callback function for the menu that is called when a menu
entry from the menu is selected.  The value passed to the callback
is determined by the value for the selected menu entry.
\end{description}

\subsubsection*{Description}

{\tt glutCreateMenu} creates a new pop-up menu and returns a unique
small integer identifier.  The range of allocated identifiers starts at
one. The menu identifier range is separate from the window identifier
range.  Implicitly, the {\em current menu} is set to the newly created
menu.  This menu identifier can be used when calling {\tt glutSetMenu}.

When the menu callback is called because a menu entry is selected for
the menu, the {\em current menu} will be implicitly set to the menu
with the selected entry before the callback is made.

\subsubsection*{X Implementation Notes}

If available, GLUT for X will take advantage of overlay planes
for implementing pop-up menus.  The use of overlay planes can
eliminate display callbacks when pop-up menus are deactivated.
The \rndx{SERVER\_OVERLAY\_VISUALS} convention \cite{kilgard93-layers} is used to
determine if overlay visuals are available.

\subsection{glutSetMenu, glutGetMenu}

\rndx{glutSetMenu} sets the {\em current menu}; \rndx{glutGetMenu} returns the
identifier of the {\em current menu}.

\subsubsection*{Usage}

\begin{verbatim}
void glutSetMenu(int menu);
int glutGetMenu(void);
\end{verbatim}
\begin{description}
\itemsep 0in
\item[{\tt menu}]
The identifier of the menu to make the {\em current menu}.
\end{description}

\subsubsection*{Description}

{\tt glutSetMenu} sets the {\em current menu}; {\tt glutGetMenu} returns the
identifier of the {\em current menu}.
If no menus exist or the previous {\em current menu} was destroyed,
{\tt glutGetMenu} returns zero.

\subsection{glutDestroyMenu}

\rndx{glutDestroyMenu} destroys the specified menu.

\subsubsection*{Usage}

\begin{verbatim}
void glutDestroyMenu(int menu);
\end{verbatim}
\begin{description}
\itemsep 0in
\item[{\tt menu}]
The identifier of the menu to destroy.
\end{description}

\subsubsection*{Description}

{\tt glutDestroyMenu} destroys the specified menu by {\tt menu}.  If {\tt menu}
was the {\em current menu}, the {\em current menu} becomes invalid and
{\tt glutGetMenu} will return zero.

When a menu is destroyed, this has no effect on any sub-menus for which the destroyed
menu has triggers.  Sub-menu triggers are by name, not reference.

\subsection{glutAddMenuEntry}

\rndx{glutAddMenuEntry} adds a menu entry to the bottom of the {\em current menu}.

\subsubsection*{Usage}
\begin{verbatim}
void glutAddMenuEntry(const char *name, int value);
\end{verbatim}
\begin{description}
\itemsep 0in
\item[{\tt name}]
ASCII character string to display in the menu entry.
\item[{\tt value}]
Value to return to the menu's callback function if the menu entry is selected.
\end{description}

\subsubsection*{Description}

{\tt glutAddMenuEntry} adds a menu entry to the bottom of the {\em current menu}.
The string {\tt name} will be displayed for the newly added menu entry.
If the menu entry is selected by the user, the menu's callback will be called
passing {\tt value} as the callback's parameter.

\subsection{glutAddSubMenu}

\rndx{glutAddSubMenu} adds a sub-menu trigger to the bottom of the {\em current menu}.

\subsubsection*{Usage}
\begin{verbatim}
void glutAddSubMenu(const char *name, int menu);
\end{verbatim}
\begin{description}
\itemsep 0in
\item[{\tt name}]
ASCII character string to display in the menu item from which to cascade the sub-menu.
\item[{\tt menu}]
Identifier of the menu to cascade from this sub-menu menu item.
\end{description}

\subsubsection*{Description}

{\tt glutAddSubMenu} adds a sub-menu trigger to the bottom of the {\em current menu}.
The string {\tt name} will be displayed for the newly added sub-menu trigger.
If the sub-menu trigger is entered, the sub-menu numbered {\tt menu} will
be cascaded, allowing sub-menu menu items to be selected.

\subsection{glutChangeToMenuEntry}

\rndx{glutChangeToMenuEntry} changes the specified menu item in the {\em current
menu} into a menu entry.

\subsubsection*{Usage}
\begin{verbatim}
void glutChangeToMenuEntry(int entry, const char *name, int value);
\end{verbatim}
\begin{description}
\itemsep 0in
\item[{\tt entry}]
Index into the menu items of the {\em current menu} (1 is the topmost
menu item).
\item[{\tt name}]
ASCII character string to display in the menu entry.
\item[{\tt value}]
Value to return to the menu's callback function if the menu entry is selected.
\end{description}

\subsubsection*{Description}

{\tt glutChangeToMenuEntry} changes the specified menu entry in the {\em current
menu} into a menu entry.  The {\tt entry} parameter determines which menu
item should be changed, with one being the topmost item.  
{\tt entry} must be between 1 and {\tt glutGet(GLUT\_MENU\_NUM\_ITEMS)}
inclusive.
The menu item to
change does not have to be a menu entry already.  The string {\tt name}
will be displayed for the newly changed menu entry.  The {\tt value}
will be returned to the menu's callback if this menu entry is selected.

\subsection{glutChangeToSubMenu}

\rndx{glutChangeToSubMenu} changes the specified menu item in the {\em
current menu} into a sub-menu trigger.

\subsubsection*{Usage}
\begin{verbatim}
void glutChangeToSubMenu(int entry, const char *name, int menu);
\end{verbatim}
\begin{description}
\itemsep 0in
\item[{\tt entry}]
Index into the menu items of the {\em current menu} (1 is the topmost
menu item).
\item[{\tt name}]
ASCII character string to display in the menu item to cascade the sub-menu from.
\item[{\tt menu}]
Identifier of the menu to cascade from this sub-menu menu item.
\end{description}

\subsubsection*{Description}

{\tt glutChangeToSubMenu} changes the specified menu item in the {\em
current menu} into a sub-menu trigger.  The {\tt entry} parameter
determines which menu item should be changed, with one being the
topmost item.  
{\tt entry} must be between 1 and {\tt glutGet(GLUT\_MENU\_NUM\_ITEMS)}
inclusive.
The menu item to change does not have to be a sub-menu
trigger already.  The string {\tt name}
will be displayed for the newly changed sub-menu trigger.
The {\tt menu} identifier names the sub-menu to cascade from the
newly added sub-menu trigger.

\subsection{glutRemoveMenuItem}

\rndx{glutRemoveMenuItem} remove the specified menu item.

\subsubsection*{Usage}
\begin{verbatim}
void glutRemoveMenuItem(int entry);
\end{verbatim}
\begin{description}
\itemsep 0in
\item[{\tt entry}]
Index into the menu items of the {\em current menu} (1 is the topmost
menu item).
\end{description}

\subsubsection*{Description}

{\tt glutRemoveMenuItem} remove the {\tt entry} menu item regardless of whether it
is a menu entry or sub-menu trigger.
{\tt entry} must be between 1 and {\tt glutGet(GLUT\_MENU\_NUM\_ITEMS)}
inclusive.
Menu items below the removed menu item are renumbered.

\subsection{glutAttachMenu, glutDetachMenu}

\rndx{glutAttachMenu} attaches a mouse button for the {\em current window} to the identifier of
the {\em current menu}; {\tt glutDetachMenu} detaches an attached mouse button from the
{\em current window}.

\subsubsection*{Usage}
\begin{verbatim}
void glutAttachMenu(int button);
void glutDetachMenu(int button);
\end{verbatim}
\begin{description}
\itemsep 0in
\item[{\tt button}]
The button to attach a menu or detach a menu.
\end{description}

\subsubsection*{Description}

{\tt glutAttachMenu} attaches a mouse button for the {\em current window} to the identifier of
the {\em current menu}; {\tt glutDetachMenu} detaches an attached mouse button from the
{\em current window}.  By attaching a menu identifier to a button, the named menu
will be popped up when the user presses the specified button.
{\tt button} should be one of {\tt GLUT\_LEFT\_BUTTON}, {\tt GLUT\_MIDDLE\_BUTTON},
and {\tt GLUT\_RIGHT\_BUTTON}.  Note that the menu is attached to the button by
identifier, not by reference.

\section{Callback Registration}

GLUT supports a number of callbacks to respond to events.  There are
three types of callbacks:  window, menu, and global.   Window callbacks
indicate when to redisplay or reshape a window, when the visibility of
the window changes, and when input is available for the window.
The menu callback is set by the {\tt glutCreateMenu} call described already.
The global callbacks manage the passing of time and menu usage.
The calling order of callbacks between different windows is undefined.

Callbacks for input events should be delivered to the window the event
occurs in.  Events should not propagate to parent windows.

\subsubsection*{X Implementation Notes}

The X GLUT implementation uses the X Input extension \cite{xinput-lib,xinput-protocol}
to support sophisticated input devices:  Spaceball, dial \& button box,
and digitizing tablet.  Because the X Input extension\index{X Input Extension} does not
mandate how particular types of devices are advertised through
the extension, it is possible GLUT for X may not
correctly support input devices that would otherwise be of the
correct type.  The X GLUT implementation will support the Silicon
Graphics Spaceball, dial \& button box, and digitizing tablet as
advertised through the X Input extension.

\subsection{glutDisplayFunc}

\rndx{glutDisplayFunc} sets the display callback for the {\em current window}.

\subsubsection*{Usage}

\begin{verbatim}
void glutDisplayFunc(void (*func)(void));
\end{verbatim}
\begin{description}
\itemsep 0in
\item[{\tt func}]
The new display callback function.
\end{description}

\subsubsection*{Description}

{\tt glutDisplayFunc} sets the display callback for the {\em current
window}.  When GLUT determines that the normal plane for the window needs
to be redisplayed, the display callback for the window is called.  Before the callback,
the {\em current window} is set to the window needing to be
redisplayed and (if no overlay display callback is registered) the {\em layer in use}
is set to the normal plane.  The display callback is called with no parameters.  The
entire normal plane region should be redisplayed in response to the callback (this
includes ancillary buffers if your program depends on their state).

GLUT determines when the display callback should be triggered based on
the window's redisplay state.  The redisplay state for a window can
be either set explicitly by calling {\tt glutPostRedisplay} or implicitly
as the result of window damage reported by the window system.
Multiple posted redisplays for a window are
coalesced by GLUT to minimize the number of display callbacks called.

When an overlay is established for a window, but there is no overlay display
callback registered, the display callback is used for redisplaying
{\em both} the overlay and normal
plane (that is, it will be called if either the redisplay
state or overlay redisplay state is set).  In this case, the {\em layer in use} is {\em not}
implicitly changed on entry to the display callback.

See {\tt glutOverlayDisplayFunc} to understand how distinct callbacks
for the overlay and normal plane of a window may be established.

When a window is created, no display callback exists for
the window. 
It is the responsibility of the programmer to install
a display callback for the window before the window is shown.
A display callback {\em must} be registered for any window that is
shown.  If a window becomes displayed without a display
callback being registered, a fatal error occurs. 
Passing {\tt NULL} to {\tt glutDisplayFunc} is illegal as of GLUT 3.0;
there is no way to ``deregister'' a display callback (though another
callback routine can always be registered).

Upon return from the display callback, the {\em normal damaged} state of the
window (returned by calling {\tt glutLayerGet(GLUT\_NORMAL\_DAMAGED)}
is cleared.  If there is no overlay display callback registered the
{\em overlay damaged} state of the window (returned by calling {\tt
glutLayerGet(GLUT\_OVERLAY\_DAMAGED)} is also cleared.

\subsection{glutOverlayDisplayFunc}

\rndx{glutOverlayDisplayFunc} sets the overlay display callback
for the {\em current window}.

\subsubsection*{Usage}

\begin{verbatim}
void glutOverlayDisplayFunc(void (*func)(void));
\end{verbatim}
\begin{description}
\itemsep 0in
\item[{\tt func}]
The new overlay display callback function.
\end{description}

\subsubsection*{Description}

{\tt glutDisplayFunc} sets the overlay display callback for the {\em current
window}.  The overlay display callback is functionally the same as the window's display callback
except that the overlay display callback is used to redisplay the window's overlay.

When GLUT determines that the overlay plane for the window needs
to be redisplayed, the overlay display callback for the window is called.  Before the callback,
the {\em current window} is set to the window needing to be
redisplayed and the {\em layer in use} is set to the overlay.  The overlay display callback
is called with no parameters.  The
entire overlay region should be redisplayed in response to the callback (this
includes ancillary buffers if your program depends on their state).

GLUT determines when the overlay display callback should be triggered based on
the window's overlay redisplay state.  The overlay redisplay state for a window can
be either set explicitly by calling {\tt glutPostOverlayRedisplay} or implicitly
as the result of window damage reported by the window system.
Multiple posted overlay redisplays for a window are
coalesced by GLUT to minimize the number of overlay display callbacks called.

Upon return from the overlay display callback, the {\em overlay damaged} state of the
window (returned by calling {\tt glutLayerGet(GLUT\_OVERLAY\_DAMAGED)}
is cleared.

The overlay display callback can be deregistered by passing {\tt NULL} to
{\tt glutOverlayDisplayFunc}.  The overlay display callback is initially {\tt NULL}
when an overlay is established.
See {\tt glutDisplayFunc} to understand how the
display callback alone is used if an overlay display callback is not registered.

\subsection{glutReshapeFunc}

\rndx{glutReshapeFunc} sets the reshape callback for the {\em current window}.

\subsubsection*{Usage}
\begin{verbatim}
void glutReshapeFunc(void (*func)(int width, int height));
\end{verbatim}
\begin{description}
\itemsep 0in
\item[{\tt func}]
The new reshape callback function.
\end{description}

\subsubsection*{Description}

{\tt glutReshapeFunc} sets the reshape callback for the {\em current window}.
The reshape callback is triggered when a window is reshaped.
A reshape callback is also triggered immediately before a window's first display callback
after a window is created or whenever an overlay for the window is established.
The {\tt width} and {\tt height} parameters of the callback
specify the new window size in pixels.  Before the callback, the {\em current
window} is set to the window that has been reshaped.

If a reshape callback is not registered for a window or {\tt NULL} is
passed to {\tt glutReshapeFunc} (to deregister a previously registered
callback), the default reshape callback is used.  This default callback
will simply call {\tt glViewport(0,0,width,height)} on the normal plane
(and on the overlay if one exists).

If an overlay is established for the window, a single reshape callback is generated.
It is the callback's responsibility to update both the normal plane and
overlay for the window (changing the {\em layer in use} as necessary).

When a top-level window is reshaped, subwindows are not reshaped.  It is
up to the GLUT program to manage the size and positions of subwindows
within a top-level window.  Still, reshape callbacks will be triggered for
subwindows when their size is changed using {\tt glutReshapeWindow}.

\subsection{glutKeyboardFunc}

\rndx{glutKeyboardFunc} sets the keyboard callback for the {\em current window}.

\subsubsection*{Usage}

\begin{verbatim}
void glutKeyboardFunc(void (*func)(unsigned char key,
                                   int x, int y));
\end{verbatim}
\begin{description}
\itemsep 0in
\item[{\tt func}]
The new keyboard callback function.
\end{description}

\subsubsection*{Description}

{\tt glutKeyboardFunc} sets the keyboard callback for the {\em current window}.
When a user types into the window, each key press generating an ASCII character
will generate a keyboard callback.  The {\tt key} callback parameter is the 
generated ASCII character.  The state of modifier keys such as Shift cannot be
determined directly; their only effect will be on the returned ASCII data.
The {\tt x} and {\tt y} callback
parameters indicate the mouse location in window relative coordinates
when the key was pressed.  When a new window is created, no keyboard
callback is initially registered, and ASCII key strokes in the window are
ignored.  
Passing {\tt NULL} to {\tt glutKeyboardFunc} disables
the generation of keyboard callbacks.

During a keyboard callback, {\tt glutGetModifiers} may be called to determine
the state of modifier keys when the keystroke generating the callback occurred.

Also, see {\tt glutSpecialFunc} for a means to detect non-ASCII key strokes.

\subsection{glutMouseFunc}

\rndx{glutMouseFunc} sets the mouse callback for the {\em current window}.

\subsubsection*{Usage}

\begin{verbatim}
void glutMouseFunc(void (*func)(int button, int state,
                                int x, int y));
\end{verbatim}
\begin{description}
\itemsep 0in
\item[{\tt func}]
The new mouse callback function.
\end{description}

\subsubsection*{Description}

{\tt glutMouseFunc} sets the mouse callback for the {\em current window}.
When a user presses and releases mouse buttons in the window, each press
and each release generates a mouse callback.  The {\tt button} parameter
is one of {\tt GLUT\_LEFT\_BUTTON}, {\tt GLUT\_MIDDLE\_BUTTON}, or
{\tt GLUT\_RIGHT\_BUTTON}.  For systems with only two mouse buttons, it may
not be possible to generate {\tt GLUT\_MIDDLE\_BUTTON} callback.  For systems
with a single mouse button, it may be possible to generate only a
{\tt GLUT\_LEFT\_BUTTON} callback.  The {\tt state} parameter is either
{\tt GLUT\_UP} or {\tt GLUT\_DOWN} indicating whether the callback
was due to a release or press respectively.  The {\tt x} and {\tt y} callback
parameters indicate the window relative coordinates when the mouse
button state changed.  If a {\tt GLUT\_DOWN} callback for a specific
button is triggered, the program can assume a {\tt GLUT\_UP} callback for the 
same button will be generated (assuming the window still has a mouse
callback registered) when the mouse button is released even if
the mouse has moved outside the window.

If a menu is attached to a button for a window, mouse callbacks will
not be generated for that button.

During a mouse callback, {\tt glutGetModifiers} may be called to determine
the state of modifier keys when the mouse event generating the
callback occurred.

Passing {\tt NULL} to {\tt glutMouseFunc} disables
the generation of mouse callbacks.

\subsection{glutMotionFunc, glutPassiveMotionFunc}

\rndx{glutMotionFunc} and {\tt glutPassiveMotionFunc} set the motion and
passive motion callbacks respectively for the {\em current window}.

\subsubsection*{Usage}
\begin{verbatim}
void glutMotionFunc(void (*func)(int x, int y));
void glutPassiveMotionFunc(void (*func)(int x, int y));
\end{verbatim}
\begin{description}
\itemsep 0in
\item[{\tt func}]
The new motion or passive motion callback function.
\end{description}

\subsubsection*{Description}

{\tt glutMotionFunc} and {\tt glutPassiveMotionFunc} set the motion and
passive motion callback respectively for the {\em current window}.
The motion callback for a window is called when the mouse moves
within the window while one or more mouse buttons are pressed.  The
passive motion callback for a window is called when the mouse moves
within the window while {\em no} mouse buttons are pressed.

The {\tt x} and {\tt y} callback parameters indicate the mouse location
in window relative coordinates.

Passing {\tt NULL} to {\tt glutMotionFunc} or {\tt glutPassiveMotionFunc}
disables the generation of the mouse or passive motion callback
respectively.

\subsection{glutVisibilityFunc}

\rndx{glutVisibilityFunc} sets the visibility callback for the {\em
current window}.

\subsubsection*{Usage}
\begin{verbatim}
void glutVisibilityFunc(void (*func)(int state));
\end{verbatim}
\begin{description}
\itemsep 0in
\item[{\tt func}]
The new visibility callback function.
\end{description}

\subsubsection*{Description}

{\tt glutVisibilityFunc} sets the visibility callback for the {\em
current window}.  The visibility callback for a window is called when
the visibility of a window changes.  The {\tt state} callback parameter
is either {\tt GLUT\_NOT\_VISIBLE} or {\tt GLUT\_VISIBLE} depending on
the current visibility of the window.  {\tt GLUT\_VISIBLE} does not
distinguish a window being totally versus partially visible.  {\tt GLUT\_NOT\_VISIBLE}
means no part of the window is visible, i.e., until the window's visibility changes,
all further rendering to the window is discarded.

GLUT considers a window visible if any pixel of the window is visible
{\em or} any pixel of any descendant window is visible on the screen.

Passing {\tt NULL} to {\tt glutVisibilityFunc}
disables the generation of the visibility callback.

If the visibility callback for a window is
disabled and later re-enabled, the visibility status of the window is
undefined; any change in window visibility will be reported, that is
if you disable a visibility callback and re-enable the callback, you
are guaranteed the next visibility change will be reported.

\subsection{glutEntryFunc}

\rndx{glutEntryFunc} sets the mouse enter/leave callback for the {\em
current window}.

\subsubsection*{Usage}
\begin{verbatim}
void glutEntryFunc(void (*func)(int state));
\end{verbatim}
\begin{description}
\itemsep 0in
\item[{\tt func}]
The new entry callback function.
\end{description}

\subsubsection*{Description}

{\tt glutEntryFunc} sets the mouse enter/leave callback for the {\em
current window}.  The {\tt state} callback parameter is either {\tt
GLUT\_LEFT} or {\tt GLUT\_ENTERED} depending on if the mouse pointer
has last left or entered the window.

Passing {\tt NULL} to {\tt glutEntryFunc}
disables the generation of the mouse enter/leave callback.

Some window systems may not generate accurate enter/leave callbacks.

\subsubsection*{X Implementation Notes}

An X implementation of GLUT should generate accurate enter/leave callbacks.

\subsection{glutSpecialFunc}

\rndx{glutSpecialFunc} sets the special keyboard callback for the {\em current window}.

\subsubsection*{Usage}
\begin{verbatim}
void glutSpecialFunc(void (*func)(int key, int x, int y));
\end{verbatim}
\begin{description}
\itemsep 0in
\item[{\tt func}]
The new special callback function.
\end{description}

\subsubsection*{Description}

{\tt glutSpecialFunc} sets the special keyboard callback for the {\em current
window}.  The special keyboard callback is triggered when keyboard
function or directional keys are pressed.  The {\tt key} callback parameter
is a {\tt GLUT\_KEY\_}* constant for the special key pressed. 
The {\tt x} and {\tt y} callback parameters indicate the mouse in window
relative coordinates when the key was pressed.  When a new window is created, no
special callback is initially registered and special key strokes in the
window are ignored.  Passing {\tt NULL} to {\tt glutSpecialFunc}
disables the generation of special callbacks.

During a special callback, {\tt glutGetModifiers} may be called to determine
the state of modifier keys when the keystroke generating the callback occurred.

An implementation should do its best to provide ways to generate all
the {\tt GLUT\_KEY\_}* special keys.  The available {\tt GLUT\_KEY\_}*
values are:
\begin{description}
\itemsep 0in
\item[{\tt GLUT\_KEY\_F1}]  F1 function key.
\item[{\tt GLUT\_KEY\_F2}]  F2 function key.
\item[{\tt GLUT\_KEY\_F3}]  F3 function key.
\item[{\tt GLUT\_KEY\_F4}]  F4 function key.
\item[{\tt GLUT\_KEY\_F5}]  F5 function key.
\item[{\tt GLUT\_KEY\_F6}]  F6 function key.
\item[{\tt GLUT\_KEY\_F7}]  F7 function key.
\item[{\tt GLUT\_KEY\_F8}]  F8 function key.
\item[{\tt GLUT\_KEY\_F9}]  F9 function key.
\item[{\tt GLUT\_KEY\_F10}]  F10 function key.
\item[{\tt GLUT\_KEY\_F11}]  F11 function key.
\item[{\tt GLUT\_KEY\_F12}]  F12 function key.
\item[{\tt GLUT\_KEY\_LEFT}]  Left directional key.
\item[{\tt GLUT\_KEY\_UP}]  Up directional key.
\item[{\tt GLUT\_KEY\_RIGHT}]  Right directional key.
\item[{\tt GLUT\_KEY\_DOWN}]  Down directional key.
\item[{\tt GLUT\_KEY\_PAGE\_UP}]  Page up directional key.
\item[{\tt GLUT\_KEY\_PAGE\_DOWN}]  Page down directional key.
\item[{\tt GLUT\_KEY\_HOME}]  Home directional key.
\item[{\tt GLUT\_KEY\_END}]  End directional key.
\item[{\tt GLUT\_KEY\_INSERT}] Inset directional key.
\end{description}
Note that the escape, backspace, and delete keys are generated as an ASCII character.

\subsection{glutSpaceballMotionFunc}

\rndx{glutSpaceballMotionFunc} sets the Spaceball motion callback for the {\em current window}.

\subsubsection*{Usage}
\begin{verbatim}
void glutSpaceballMotionFunc(void (*func)(int x, int y, int z));
\end{verbatim}
\begin{description}
\itemsep 0in
\item[{\tt func}]
The new spaceball motion callback function.
\end{description}

\subsubsection*{Description}

{\tt glutSpaceballMotionFunc} sets the Spaceball motion callback for the {\em
current window}.  The Spaceball motion callback for a window is called when the
window has Spaceball input focus (normally, when the mouse is in the window)
and the user generates Spaceball translations.  The {\tt x}, {\tt y}, and {\tt z}
callback parameters indicate the translations along the X, Y, and Z axes.
The callback parameters are normalized to be within the range of -1000 to 1000
inclusive.

Registering a Spaceball motion callback when a Spaceball device is not available
has no effect and is not an error.  In this case, no Spaceball motion callbacks
will be generated.

Passing {\tt NULL} to {\tt glutSpaceballMotionFunc} disables the generation of
Spaceball motion callbacks.  When a new window is created, no
Spaceball motion callback is initially registered.

\subsection{glutSpaceballRotateFunc}

\rndx{glutSpaceballRotateFunc} sets the Spaceball rotation callback
for the {\em current window}.

\subsubsection*{Usage}
\begin{verbatim}
void glutSpaceballRotateFunc(void (*func)(int x, int y, int z));
\end{verbatim}
\begin{description}
\itemsep 0in
\item[{\tt func}]
The new spaceball rotate callback function.
\end{description}

\subsubsection*{Description}

{\tt glutSpaceballRotateFunc} sets the Spaceball rotate callback for the {\em
current window}.  The Spaceball rotate callback for a window is called when the
window has Spaceball input focus (normally, when the mouse is in the window)
and the user generates Spaceball rotations.  The {\tt x}, {\tt y}, and {\tt z}
callback parameters indicate the rotation along the X, Y, and Z axes.
The callback parameters are normalized to be within the range of -1800 to 1800
inclusive.

Registering a Spaceball rotate callback when a Spaceball device is not available
is ineffectual and not an error.  In this case, no Spaceball rotate callbacks
will be generated.

Passing {\tt NULL} to {\tt glutSpaceballRotateFunc} disables the generation of
Spaceball rotate callbacks.  When a new window is created, no
Spaceball rotate callback is initially registered.

\subsection{glutSpaceballButtonFunc}

\rndx{glutSpaceballButtonFunc} sets the Spaceball button callback for
the {\em current window}.

\subsubsection*{Usage}
\begin{verbatim}
void glutSpaceballButtonFunc(void (*func)(int button, int state));
\end{verbatim}
\begin{description}
\itemsep 0in
\item[{\tt func}]
The new spaceball button callback function.
\end{description}

\subsubsection*{Description}

{\tt glutSpaceballButtonFunc} sets the Spaceball button callback for the {\em
current window}.  The Spaceball button callback for a window is called when the
window has Spaceball input focus (normally, when the mouse is in the window)
and the user generates Spaceball button presses.  The {\tt button} parameter
will be the button number (starting at one).  The number of available Spaceball
buttons can be determined with {\tt glutDeviceGet(GLUT\_NUM\_SPACEBALL\_BUTTONS)}.
The {\tt state} is either {\tt GLUT\_UP} or {\tt GLUT\_DOWN} indicating whether
the callback was due to a release or press respectively.

Registering a Spaceball button callback when a Spaceball device is not available
is ineffectual and not an error.  In this case, no Spaceball button callbacks
will be generated.

Passing {\tt NULL} to {\tt glutSpaceballButtonFunc} disables the generation of
Spaceball button callbacks.  When a new window is created, no
Spaceball button callback is initially registered.

\subsection{glutButtonBoxFunc}

\rndx{glutButtonBoxFunc} sets the dial \& button box button callback for
the {\em current window}.

\subsubsection*{Usage}
\begin{verbatim}
void glutButtonBoxFunc(void (*func)(int button, int state));
\end{verbatim}
\begin{description}
\itemsep 0in
\item[{\tt func}]
The new button box callback function.
\end{description}

\subsubsection*{Description}

{\tt glutButtonBoxFunc} sets the dial \& button box button callback for the {\em
current window}.  The dial \& button box button callback for a window is called when the
window has dial \& button box input focus (normally, when the mouse is in the window)
and the user generates dial \& button box button presses.  The {\tt button} parameter
will be the button number (starting at one).  The number of available dial \&
button box
buttons can be determined with {\tt glutDeviceGet(GLUT\_NUM\_BUTTON\_BOX\_BUTTONS)}.
The {\tt state} is either {\tt GLUT\_UP} or {\tt GLUT\_DOWN} indicating whether
the callback was due to a release or press respectively.

Registering a dial \& button box button callback when a dial \& button box
device is not available
is ineffectual and not an error.  In this case, no dial \& button box button callbacks
will be generated.

Passing {\tt NULL} to {\tt glutButtonBoxFunc} disables the generation of
dial \& button box button callbacks.  When a new window is created, no
dial \& button box button callback is initially registered.

\subsection{glutDialsFunc}

\rndx{glutDialsFunc} sets the dial \& button box dials callback for
the {\em current window}.

\subsubsection*{Usage}
\begin{verbatim}
void glutDialsFunc(void (*func)(int dial, int value));
\end{verbatim}
\begin{description}
\itemsep 0in
\item[{\tt func}]
The new dials callback function.
\end{description}

\subsubsection*{Description}

{\tt glutDialsFunc} sets the dial \& button box dials callback for the {\em
current window}.  The dial \& button box dials callback for a window is called when the
window has dial \& button box input focus (normally, when the mouse is in the window)
and the user generates dial \& button box dial changes.  The {\tt dial} parameter
will be the dial number (starting at one).  The number of available dial \&
button box
dials can be determined with {\tt glutDeviceGet(GLUT\_NUM\_DIALS)}.
The {\tt value} measures the absolute rotation in degrees.
Dial values do not ``roll over'' with each complete rotation but
continue to accumulate degrees (until the {\tt int} dial value
overflows).

Registering a dial \& button box dials callback when a dial \& button box
device is not available
is ineffectual and not an error.  In this case, no dial \& button box dials callbacks
will be generated.

Passing {\tt NULL} to {\tt glutDialsFunc} disables the generation of
dial \& button box dials callbacks.  When a new window is created, no
dial \& button box dials callback is initially registered.

\subsection{glutTabletMotionFunc}

\rndx{glutTabletMotionFunc} sets the special keyboard callback for the {\em current window}.

\subsubsection*{Usage}
\begin{verbatim}
void glutTabletMotionFunc(void (*func)(int x, int y));
\end{verbatim}
\begin{description}
\itemsep 0in
\item[{\tt func}]
The new tablet motion callback function.
\end{description}

\subsubsection*{Description}

{\tt glutTabletMotionFunc} sets the tablet motion callback for the {\em
current window}.  The tablet motion callback for a window is called when the
window has tablet input focus (normally, when the mouse is in the window)
and the user generates tablet motion.  The {\tt x} and {\tt y}
callback parameters indicate the absolute position of the tablet ``puck'' on
the tablet.
The callback parameters are normalized to be within the range of 0 to 2000
inclusive.

Registering a tablet motion callback when a tablet device is not available
is ineffectual and not an error.  In this case, no tablet motion callbacks
will be generated.

Passing {\tt NULL} to {\tt glutTabletMotionFunc} disables the generation of
tablet motion callbacks.  When a new window is created, no
tablet motion callback is initially registered.

\subsection{glutTabletButtonFunc}

\rndx{glutTabletButtonFunc} sets the special keyboard callback for the {\em current window}.

\subsubsection*{Usage}
\begin{verbatim}
void glutTabletButtonFunc(void (*func)(int button, int state,
                          int x, int y));
\end{verbatim}
\begin{description}
\itemsep 0in
\item[{\tt func}]
The new tablet button callback function.
\end{description}

\subsubsection*{Description}

{\tt glutTabletButtonFunc} sets the tablet button callback for the {\em
current window}.  The tablet button callback for a window is called when the
window has tablet input focus (normally, when the mouse is in the window)
and the user generates tablet button presses.  The {\tt button} parameter
will be the button number (starting at one).  The number of available tablet
buttons can be determined with {\tt glutDeviceGet(GLUT\_NUM\_TABLET\_BUTTONS)}.
The {\tt state} is either {\tt GLUT\_UP} or {\tt GLUT\_DOWN} indicating whether
the callback was due to a release or press respectively.
The {\tt x} and {\tt y} callback
parameters indicate the window relative coordinates when the tablet
button state changed.

Registering a tablet button callback when a tablet device is not available
is ineffectual and not an error.  In this case, no tablet button callbacks
will be generated.

Passing {\tt NULL} to {\tt glutTabletButtonFunc} disables the generation of
tablet button callbacks.  When a new window is created, no
tablet button callback is initially registered.

\subsection{glutMenuStatusFunc}

\rndx{glutMenuStatusFunc} sets the global menu status callback.

\subsubsection*{Usage}
\begin{verbatim}
void glutMenuStatusFunc(void (*func)(int status, int x, int y));
void glutMenuStateFunc(void (*func)(int status));
\end{verbatim}
\begin{description}
\itemsep 0in
\item[{\tt func}]
The new menu status (or state) callback function.
\end{description}

\subsubsection*{Description}

{\tt glutMenuStatusFunc} sets the global menu status callback so a GLUT
program can determine when a menu is in use or not.  When a menu status
callback is registered, it will be called with the value {\tt
GLUT\_MENU\_IN\_USE} for its {\tt value} parameter when pop-up menus are
in use by the user; and the callback will be called with the value {\tt
GLUT\_MENU\_NOT\_IN\_USE} for its {\tt status} parameter when pop-up menus
are no longer in use.  The {\tt x} and {\tt y} parameters indicate the
location in window coordinates of the button press that caused the menu
to go into use, or the location where the menu was released (may be
outside the window).  The {\tt func} parameter names the callback
function.  Other callbacks continue to operate (except mouse motion 
callbacks) when pop-up menus are in use so the menu status callback
allows a program to suspend animation or other tasks when menus are in
use.  The cascading and unmapping of sub-menus from an initial pop-up
menu does not generate menu status callbacks.  There is
a single menu status callback for GLUT.

When the menu status callback is called, the {\em current menu} will be
set to the initial pop-up menu in both the {\tt GLUT\_MENU\_IN\_USE} and
{\tt GLUT\_MENU\_NOT\_IN\_USE} cases.  The {\em current window} will be
set to the window from which the initial menu was popped up from,
also in both cases.

Passing {\tt NULL} to {\tt glutMenuStatusFunc}
disables the generation of the menu status callback.

{\tt glutMenuStateFunc} is a deprecated version of the {\tt glutMenuStatusFunc}
routine.  The only difference is {\tt glutMenuStateFunc} callback prototype
does not deliver the two additional {\tt x} and {\tt y} coordinates.

\subsection{glutIdleFunc}

\rndx{glutIdleFunc} sets the global idle callback.

\subsubsection*{Usage}
\begin{verbatim}
void glutIdleFunc(void (*func)(void));
\end{verbatim}
\begin{description}
\itemsep 0in
\item[{\tt func}]
The new idle callback function.
\end{description}

\subsubsection*{Description}

{\tt glutIdleFunc} sets the global idle callback to be {\tt func}
so a GLUT program
can perform background processing tasks or continuous animation when
window system events are not being received.  If enabled, the idle
callback is continuously called when events are not being
received.  The callback routine has no parameters.  The {\em current
window} and {\em current menu} will not be changed before the
idle callback.  Programs with multiple windows and/or menus should
explicitly set the {\em current window} and/or {\em current menu}
and not rely on its current setting.

The amount of computation and rendering done in an idle callback
should be minimized to avoid affecting the program's interactive
response.  In general, not more than a single frame of rendering
should be done in an idle callback.

Passing {\tt NULL} to {\tt glutIdleFunc}
disables the generation of the idle callback.

\subsection{glutTimerFunc}

\rndx{glutTimerFunc} registers a timer callback to be triggered
in a specified number of milliseconds.

\subsubsection*{Usage}
\begin{verbatim}
void glutTimerFunc(unsigned int msecs,
                   void (*func)(int value), int value);
\end{verbatim}
\begin{description}
\itemsep 0in
\item[{\tt msecs}]
Number of milliseconds to pass before calling the callback.
\item[{\tt func}]
The timer callback function.
\item[{\tt value}]
Integer value to pass to the timer callback.
\end{description}

\subsubsection*{Description}

{\tt glutTimerFunc} registers the timer callback {\tt func} to be triggered
in at least {\tt msecs} milliseconds.  The {\tt value} parameter to the timer
callback will be the value of the {\tt value} parameter to {\tt glutTimerFunc}.
Multiple timer callbacks at same or differing times may be registered
simultaneously.

The number of milliseconds is a lower bound on the time before the
callback is generated.  GLUT attempts to deliver the timer callback
as soon as possible after the expiration of the callback's time interval.

There is no support for canceling a registered callback.  Instead,
ignore a callback based on its {\tt value} parameter when it is triggered.

\section{Color Index Colormap Management}

OpenGL supports both RGBA and color index rendering.  The RGBA mode is generally
preferable to color index because more OpenGL rendering capabilities are
available and color index mode requires the loading of colormap entries.

The GLUT color index routines are used to write and read entries in a window's
color index colormap.  Every GLUT color index window has its own logical color
index colormap.  The size of a window's colormap can be determined by calling
{\tt glutGet(GLUT\_WINDOW\_COLORMAP\_SIZE)}.

GLUT color index windows within a program can attempt to share colormap resources
by copying a single color index colormap to multiple windows using {\tt glutCopyColormap}.
If possible GLUT will attempt to share the actual colormap.  While copying
colormaps using {\tt glutCopyColormap} can potentially allow sharing of
physical colormap resources, logically each window has its own colormap.  So
changing a copied colormap of a window will force the duplication of the colormap.
For this reason, color index programs should generally load a single color index
colormap, copy it to all color index windows within the program, and then not
modify any colormap cells.

Use of multiple colormaps is likely to result in
colormap installation problems where some windows are
displayed with an incorrect colormap due to limitations on colormap
resources.

\subsection{glutSetColor}

\rndx{glutSetColor} sets the color of a colormap entry in the {\em layer of use}
for the {\em current window}.

\subsubsection*{Usage}
\begin{verbatim}
void glutSetColor(int cell,
                  GLfloat red, GLfloat green, GLfloat blue);
\end{verbatim}
\begin{description}
\itemsep 0in
\item[{\tt cell}]
Color cell index (starting at zero).
\item[{\tt red}]
Red intensity (clamped between 0.0 and 1.0 inclusive).
\item[{\tt green}]
Green intensity (clamped between 0.0 and 1.0 inclusive).
\item[{\tt blue}]
Blue intensity (clamped between 0.0 and 1.0 inclusive).
\end{description}

\subsubsection*{Description}

Sets the {\tt cell} color index colormap entry of the {\em current window}'s logical
colormap for the {\em layer in use} with the color specified by {\tt red}, {\tt green}, and {\tt blue}.
The {\em layer in use} of the {\em current window} should be a color index window.
{\tt cell} should be zero or greater and less than the total number of colormap entries
for the window.
If the {\em layer in use}'s colormap was copied by reference,
a {\tt glutSetColor} call will force the duplication of the colormap.
Do not attempt to set the color of an overlay's transparent index.

\subsection{glutGetColor}

\rndx{glutGetColor} retrieves a red, green, or blue component for a given
color index colormap entry for the {\em layer in use}'s logical colormap
for the {\em current window}.

\subsubsection*{Usage}
\begin{verbatim}
GLfloat glutGetColor(int cell, int component);
\end{verbatim}
\begin{description}
\itemsep 0in
\item[{\tt cell}]
Color cell index (starting at zero).
\item[{\tt component}]
One of {\tt GLUT\_RED}, {\tt GLUT\_GREEN}, or {\tt GLUT\_BLUE}.
\end{description}

\subsubsection*{Description}

{\tt glutGetColor} retrieves a red, green, or blue component for a given
color index colormap entry for the {\em current window}'s logical colormap.
The {\em current window} should be a color index window.
{\tt cell} should be zero or greater and less than the total number of colormap entries
for the window.
For valid color indices, the value returned is a floating point value between 0.0
and 1.0 inclusive.  {\tt glutGetColor} will return -1.0 if the color
index specified is an overlay's transparent index, less than zero, or
greater or equal to the value returned by {\tt
glutGet(GLUT\_WINDOW\_COLORMAP\_SIZE)}, that is if the color index is
transparent or outside the valid range of color indices.

\subsection{glutCopyColormap}

\rndx{glutCopyColormap} copies the logical
colormap for the {\em layer in use} from a specified window to the {\em current window}.

\subsubsection*{Usage}
\begin{verbatim}
void glutCopyColormap(int win);
\end{verbatim}
\begin{description}
\itemsep 0in
\item[{\tt win}]
The identifier of the window to copy the logical colormap from.
\end{description}

\subsubsection*{Description}

{\tt glutCopyColormap} copies (lazily if possible to promote sharing) the logical
colormap from a specified window to the {\em current window}'s {\em layer in use}.
The copy will be from the normal plane to the normal plane; or from the overlay
to the overlay (never across different layers).  Once a colormap has been
copied, avoid setting cells in the colormap with {\tt glutSetColor} since that
will force an actual copy of the colormap if it was previously copied by reference.
{\tt glutCopyColormap} should only be called when both the {\em current window} and
the {\tt win} window are color index windows.

\section{State Retrieval}

GLUT maintains a considerable amount of programmer visible state.
Some (but not all) of this state may be directly retrieved.

\subsection{glutGet}

\rndx{glutGet} retrieves simple GLUT state represented by integers.

\subsubsection*{Usage}
\begin{verbatim}
int glutGet(GLenum state);
\end{verbatim}
\begin{description}
\item[{\tt state}]  Name of state to retrieve.
\itemsep 0in
\end{description}

\begin{description}
\itemsep 0in
\item[{\tt GLUT\_WINDOW\_X}]
X location in pixels (relative to the screen origin) of the {\em current window}.
\item[{\tt GLUT\_WINDOW\_Y}]
Y location in pixels (relative to the screen origin) of the {\em current window}.
\item[{\tt GLUT\_WINDOW\_WIDTH}]
Width in pixels of the {\em current window}.
\item[{\tt GLUT\_WINDOW\_HEIGHT}]
Height in pixels of the {\em current window}.
\item[{\tt GLUT\_WINDOW\_BUFFER\_SIZE}]
Total number of bits for {\em current window}'s color buffer.
For an RGBA window, this is the sum of {\tt \tt GLUT\_WINDOW\_RED\_SIZE},
{\tt GLUT\_WINDOW\_GREEN\_SIZE}, {\tt GLUT\_WINDOW\_BLUE\_SIZE},
and {\tt GLUT\_WINDOW\_ALPHA\_SIZE}.  For color index windows, this
is the number of bits for color indices.
\item[{\tt GLUT\_WINDOW\_STENCIL\_SIZE}]
Number of bits in the {\em current window}'s stencil buffer.
\item[{\tt GLUT\_WINDOW\_DEPTH\_SIZE}]
Number of bits in the {\em current window}'s depth buffer.
\item[{\tt GLUT\_WINDOW\_RED\_SIZE}]
Number of bits of red stored the {\em current window}'s color buffer.
Zero if the window is color index.
\item[{\tt GLUT\_WINDOW\_GREEN\_SIZE}]
Number of bits of green stored the {\em current window}'s color buffer.
Zero if the window is color index.
\item[{\tt GLUT\_WINDOW\_BLUE\_SIZE}]
Number of bits of blue stored the {\em current window}'s color buffer.
Zero if the window is color index.
\item[{\tt GLUT\_WINDOW\_ALPHA\_SIZE}]
Number of bits of alpha stored the {\em current window}'s color buffer.
Zero if the window is color index.
\item[{\tt GLUT\_WINDOW\_ACCUM\_RED\_SIZE}]
Number of bits of red stored in the {\em current window}'s accumulation buffer.
Zero if the window is color index.
\item[{\tt GLUT\_WINDOW\_ACCUM\_GREEN\_SIZE}]
Number of bits of green stored in the {\em current window}'s accumulation buffer.
Zero if the window is color index.
\item[{\tt GLUT\_WINDOW\_ACCUM\_BLUE\_SIZE}]
Number of bits of blue stored in the {\em current window}'s accumulation buffer.
Zero if the window is color index.
\item[{\tt GLUT\_WINDOW\_ACCUM\_ALPHA\_SIZE}]
Number of bits of alpha stored in the {\em current window}'s accumulation buffer.
Zero if the window is color index.
\item[{\tt GLUT\_WINDOW\_DOUBLEBUFFER}]
One if the {\em current window} is double buffered, zero otherwise.
\item[{\tt GLUT\_WINDOW\_RGBA}]
One if the {\em current window} is RGBA mode, zero otherwise (i.e., color index).
\item[{\tt GLUT\_WINDOW\_PARENT}]
The window number of the {\em current window}'s parent; zero if the window is a top-level window.
\item[{\tt GLUT\_WINDOW\_NUM\_CHILDREN}]
The number of subwindows the {\em current window} has (not counting children of children).
\item[{\tt GLUT\_WINDOW\_COLORMAP\_SIZE}]
Size of {\em current window}'s color index colormap; zero for RGBA color model windows.
\item[{\tt GLUT\_WINDOW\_NUM\_SAMPLES}]
Number of samples for multisampling for the {\em current window}.
\item[{\tt GLUT\_WINDOW\_STEREO}]
One if the {\em current window} is stereo, zero otherwise.
\item[{\tt GLUT\_WINDOW\_CURSOR}]
Current cursor for the {\em current window}.
\item[{\tt GLUT\_SCREEN\_WIDTH}]
Width of the screen in pixels.
Zero indicates the width is unknown or not available.
\item[{\tt GLUT\_SCREEN\_HEIGHT}]
Height of the screen in pixels.
Zero indicates the height is unknown or not available.
\item[{\tt GLUT\_SCREEN\_WIDTH\_MM}]
Width of the screen in millimeters.
Zero indicates the width is unknown or not available.
\item[{\tt GLUT\_SCREEN\_HEIGHT\_MM}]
Height of the screen in millimeters.
Zero indicates the height is unknown or not available.
\item[{\tt GLUT\_MENU\_NUM\_ITEMS}]
Number of menu items in the {\em current menu}.
\item[{\tt GLUT\_DISPLAY\_MODE\_POSSIBLE}]
Whether the {\em current display mode} is supported or not.
\item[{\tt GLUT\_INIT\_DISPLAY\_MODE}]
The {\em initial display mode} bit mask.
\item[{\tt GLUT\_INIT\_WINDOW\_X}]
The X value of the {\em initial window position}.
\item[{\tt GLUT\_INIT\_WINDOW\_Y}]
The Y value of the {\em initial window position}.
\item[{\tt GLUT\_INIT\_WINDOW\_WIDTH}]
The width value of the {\em initial window size}.
\item[{\tt GLUT\_INIT\_WINDOW\_HEIGHT}]
The height value of the {\em initial window size}.
\item[{\tt GLUT\_ELAPSED\_TIME}]
Number of milliseconds since {\tt glutInit} called (or first call
to {\tt glutGet(GLUT\_ELAPSED\_TIME)}).
\end{description}

\subsubsection*{Description}

{\tt glutGet} retrieves simple GLUT state represented by integers.
The {\tt state} parameter determines what type of state to
return.  Window capability state is returned for the {\em layer in use}.
GLUT state names beginning with {\tt GLUT\_WINDOW\_}
return state for the {\em current window}.  GLUT state names beginning
with {\tt GLUT\_MENU\_} return state for the {\em current menu}.
Other GLUT state names return global state.
Requesting state for an invalid GLUT state name returns negative one.

\subsection{glutLayerGet}

\rndx{glutLayerGet} retrieves GLUT state pertaining to the layers of the
{\em current window}.

\subsubsection*{Usage}
\begin{verbatim}
int glutLayerGet(GLenum info);
\end{verbatim}
\begin{description}
\item[{\tt info}]  Name of device information to retrieve.
\itemsep 0in
\end{description}

\begin{description}
\itemsep 0in
\item[{\tt GLUT\_OVERLAY\_POSSIBLE}]
Whether an overlay could be established for the {\em current window} given the
current {\em initial display mode}.  If false, {\tt glutEstablishOverlay}
will fail with a fatal error if called.
\item[{\tt GLUT\_LAYER\_IN\_USE}]
Either {\tt GLUT\_NORMAL} or {\tt GLUT\_OVERLAY} depending on whether the normal
plane or overlay is the {\em layer in use}.
\item[{\tt GLUT\_HAS\_OVERLAY}]
If the {\em current window} has an overlay established.
\item[{\tt GLUT\_TRANSPARENT\_INDEX}]
The transparent color index of the overlay of the {\em current window}; negative
one is returned if no overlay is in use.
\item[{\tt GLUT\_NORMAL\_DAMAGED}]
True if the normal plane of the {\em current window} has damaged (by window system
activity) since the last display callback was triggered.  
Calling {\tt glutPostRedisplay}
will not set this true.
\item[{\tt GLUT\_OVERLAY\_DAMAGED}]
True if the overlay plane of the {\em current window} has damaged (by window system
activity) since the last display callback was triggered.
Calling {\tt glutPostRedisplay} or {\tt glutPostOverlayRedisplay}
will not set this true.  Negative one is returned if no overlay is in use.
\end{description}

\subsubsection*{Description}

{\tt glutLayerGet} retrieves GLUT layer information for the {\em current window}
represented by integers.
The {\tt info} parameter determines what type of layer information to return.

\subsection{glutDeviceGet}

\rndx{glutDeviceGet} retrieves GLUT device information represented by integers.

\subsubsection*{Usage}
\begin{verbatim}
int glutDeviceGet(GLenum info);
\end{verbatim}
\begin{description}
\item[{\tt info}]  Name of device information to retrieve.
\itemsep 0in
\end{description}

\begin{description}
\itemsep 0in
\item[{\tt GLUT\_HAS\_KEYBOARD}]
Non-zero if a keyboard is available; zero if not available.  For most GLUT implementations,
a keyboard can be assumed.
\item[{\tt GLUT\_HAS\_MOUSE}]
Non-zero if a mouse is available; zero if not available.  For most GLUT implementations,
a keyboard can be assumed.
\item[{\tt GLUT\_HAS\_SPACEBALL}]
Non-zero if a Spaceball is available; zero if not available.
\item[{\tt GLUT\_HAS\_DIAL\_AND\_BUTTON\_BOX}]
Non-zero if a dial \& button box is available; zero if not available.
\item[{\tt GLUT\_HAS\_TABLET}]
Non-zero if a tablet is available; zero if not available.
\item[{\tt GLUT\_NUM\_MOUSE\_BUTTONS}]
Number of buttons supported by the mouse.  If no mouse is supported, zero
is returned.
\item[{\tt GLUT\_NUM\_SPACEBALL\_BUTTONS}]
Number of buttons supported by the Spaceball.  If no Spaceball is supported,
zero is returned.
\item[{\tt GLUT\_NUM\_BUTTON\_BOX\_BUTTONS}]
Number of buttons supported by the dial \& button box device.  If no dials
\& button box device is supported, zero is returned.
\item[{\tt GLUT\_NUM\_DIALS}]
Number of dials supported by the dial \& button box device.  If no dials
\& button box device is supported, zero is returned.
\item[{\tt GLUT\_NUM\_TABLET\_BUTTONS}]
Number of buttons supported by the tablet.  If no tablet is supported,
zero is returned.
\end{description}

\subsubsection*{Description}

{\tt glutDeviceGet} retrieves GLUT device information represented by integers.
The {\tt info} parameter determines what type of device information to return.
Requesting device information for an invalid GLUT device information name
returns negative one.

\subsection{glutGetModifiers}

\rndx{glutGetModifiers} returns the modifier key state when certain callbacks
were generated.

\subsubsection*{Usage}
\begin{verbatim}
int glutGetModifiers(void);
\end{verbatim}

\begin{description}
\itemsep 0in
\item[{\tt GLUT\_ACTIVE\_SHIFT}]  Set if the Shift modifier or Caps Lock is active.
\item[{\tt GLUT\_ACTIVE\_CTRL}]  Set if the Ctrl modifier is active.
\item[{\tt GLUT\_ACTIVE\_ALT}]  Set if the Alt modifier is active.
\end{description}

\subsubsection*{Description}

{\tt glutGetModifiers} returns the modifier key state at the time the input event
for a keyboard,
special, or mouse callback is generated.  This routine may only be
called while a keyboard, special, or mouse callback is being handled.
The window system is permitted to intercept window system defined modifier key
strokes or mouse buttons, in which case, no GLUT callback will be
generated.  This interception will be independent of use of {\tt
glutGetModifiers}.

\subsection{glutExtensionSupported}

\rndx{glutExtensionSupported} helps to easily determine whether a given
OpenGL extension is supported.

\subsubsection*{Usage}
\begin{verbatim}
int glutExtensionSupported(const char *extension);
\end{verbatim}
\begin{description}
\itemsep 0in
\item[{\tt extension}]
Name of OpenGL extension.
\end{description}

\subsubsection*{Description}

{\tt glutExtensionSupported} helps to easily determine whether a given
OpenGL extension is supported or not.  The {\tt extension} parameter
names the extension to query.  The supported extensions can also be
determined with {\tt glGetString(GL\_EXTENSIONS)}, but {\tt glutExtensionSupported}
does the correct parsing of the returned string.

{\tt glutExtensionSupported} returns non-zero if the extension is supported,
zero if not supported.

There must be a valid {\em current window} to call {\tt glutExtensionSupported}.

{\tt glutExtensionSupported} only returns information about OpenGL
extensions only.  This means window system dependent extensions (for
example, GLX extensions) are not reported by {\tt
glutExtensionSupported}.

\section{Font Rendering}

GLUT supports two type of font rendering:  stroke fonts,
meaning each character is rendered as a set of line segments;
and bitmap fonts, where each character
is a bitmap generated with {\tt glBitmap}.  Stroke fonts have
the advantage that because they are geometry, they can
be arbitrarily scale and rendered.  Bitmap fonts are
less flexible since they are rendered as bitmaps but are
usually faster than stroke fonts.

\subsection{glutBitmapCharacter}

\rndx{glutBitmapCharacter} renders a bitmap character using OpenGL.

\subsubsection*{Usage}
\begin{verbatim}
void glutBitmapCharacter(void *font, int character);
\end{verbatim}
\begin{description}
\itemsep 0in
\item[{\tt font}]
Bitmap font to use.
\item[{\tt character}]
Character to render (not confined to 8 bits).
\end{description}

\subsubsection*{Description}

Without using any display lists, {\tt glutBitmapCharacter} renders the {\tt character}
in the named bitmap {\tt font}.  The available fonts are:
\begin{description}
\item[{{\tt GLUT\_BITMAP\_8\_BY\_13}}]
A fixed width font with every character fitting in an 8 by 13 pixel
rectangle.  The exact bitmaps to be used is defined by the standard X glyph
bitmaps for the X font named:

{\tt -misc-fixed-medium-r-normal--13-120-75-75-C-80-iso8859-1}

\item[{{\tt GLUT\_BITMAP\_9\_BY\_15}}]
A fixed width font with every character fitting in an 9 by 15 pixel
rectangle.  The exact bitmaps to be used is defined by the standard X glyph
bitmaps for the X font named:

{\tt -misc-fixed-medium-r-normal--15-140-75-75-C-90-iso8859-1}

\item[{{\tt GLUT\_BITMAP\_TIMES\_ROMAN\_10}}]
A 10-point proportional spaced
Times Roman font.  The exact bitmaps to be used is defined by the
standard X glyph bitmaps for the X font named:

{\tt -adobe-times-medium-r-normal--10-100-75-75-p-54-iso8859-1}

\item[{{\tt GLUT\_BITMAP\_TIMES\_ROMAN\_24}}]
A 24-point proportional spaced
Times Roman font.  The exact bitmaps to be used is defined by the
standard X glyph bitmaps for the X font named:

{\tt -adobe-times-medium-r-normal--24-240-75-75-p-124-iso8859-1}

\item[{{\tt GLUT\_BITMAP\_HELVETICA\_10}}]
A 10-point proportional spaced
Helvetica font.  The exact bitmaps to be used is defined by the
standard X glyph bitmaps for the X font named:

{\tt -adobe-helvetica-medium-r-normal--10-100-75-75-p-56-iso8859-1}

\item[{{\tt GLUT\_BITMAP\_HELVETICA\_12}}]
A 12-point proportional spaced
Helvetica font.  The exact bitmaps to be used is defined by the
standard X glyph bitmaps for the X font named:

{\tt -adobe-helvetica-medium-r-normal--12-120-75-75-p-67-iso8859-1}

\item[{{\tt GLUT\_BITMAP\_HELVETICA\_18}}]
A 18-point proportional spaced
Helvetica font.  The exact bitmaps to be used is defined by the
standard X glyph bitmaps for the X font named:

{\tt -adobe-helvetica-medium-r-normal--18-180-75-75-p-98-iso8859-1}

\end{description}
Rendering a nonexistent character has no effect.  {\tt glutBitmapCharacter} automatically
sets the OpenGL unpack pixel storage modes it needs appropriately and saves and restores
the previous modes before returning.  The generated call to {\tt glBitmap} will
adjust the current raster position based on the width of the character.

\subsection{glutBitmapWidth}

\rndx{glutBitmapWidth} returns the width of a bitmap character.

\subsubsection*{Usage}
\begin{verbatim}
int glutBitmapWidth(GLUTbitmapFont font, int character);
\end{verbatim}
\begin{description}
\itemsep 0in
\item[{\tt font}]
Bitmap font to use.
\item[{\tt character}]
Character to return width of (not confined to 8 bits).
\end{description}

\subsubsection*{Description}

\rndx{glutBitmapWidth} returns the width in pixels of a bitmap character
in a supported bitmap font.  While the width of characters in a font
may vary (though fixed width fonts do not vary), the maximum height
characteristics of a particular font are fixed.

\subsection{glutStrokeCharacter}

\rndx{glutStrokeCharacter} renders a stroke character using OpenGL.

\subsubsection*{Usage}
\begin{verbatim}
void glutStrokeCharacter(void *font, int character);
\end{verbatim}
\begin{description}
\itemsep 0in
\item[{\tt font}]
Stroke font to use.
\item[{\tt character}]
Character to render (not confined to 8 bits).
\end{description}

\subsubsection*{Description}

Without using any display lists, {\tt glutStrokeCharacter} renders the {\tt character}
in the named stroke {\tt font}.  The available fonts are:
\begin{description}
\item[{{\tt GLUT\_STROKE\_ROMAN}}]
A proportionally spaced Roman Simplex font
for ASCII characters 32 through 127.  The maximum top character in the font
is 119.05 units; the bottom descends 33.33 units.

\item[{{\tt GLUT\_STROKE\_MONO\_ROMAN}}]
A mono-spaced spaced Roman Simplex font (same characters as {\tt GLUT\_STROKE\_ROMAN})
for ASCII characters 32 through 127.  The maximum top character in the font
is 119.05 units; the bottom descends 33.33 units.  Each character is 104.76
units wide.
\end{description}

Rendering a nonexistent character has no effect.  A {\tt glTranslatef} is used
to translate the current model view matrix to advance the width of the
character.

\subsection{glutStrokeWidth}

\rndx{glutStrokeWidth} returns the width of a stroke character.

\subsubsection*{Usage}
\begin{verbatim}
int glutStrokeWidth(GLUTstrokeFont font, int character);
\end{verbatim}
\begin{description}
\itemsep 0in
\item[{\tt font}]
Stroke font to use.
\item[{\tt character}]
Character to return width of (not confined to 8 bits).
\end{description}

\subsubsection*{Description}

{\tt glutStrokeWidth} returns the width in pixels of a stroke character
in a supported stroke font.  While the width of characters in a font
may vary (though fixed width fonts do not vary), the maximum height
characteristics of a particular font are fixed.

\section{Geometric Object Rendering}

GLUT includes a number of routines for generating easily recognizable 3D geometric
objects.  These routines reflect functionality available in the {\tt aux} toolkit described in the {\em OpenGL Programmer's Guide}

and are included in GLUT to allow the construction of simple GLUT programs that
render recognizable objects.  These routines can be implemented as pure OpenGL
rendering routines.
The routines do {\em not} generate display lists for the objects they create.

The routines generate normals appropriate for lighting but do not generate
texture coordinates (except for the teapot).

\subsection{glutSolidSphere, glutWireSphere}

\rndx{glutSolidSphere} and \rndx{glutWireSphere} render a solid or wireframe
sphere respectively.

\subsubsection*{Usage}
\begin{verbatim}
void glutSolidSphere(GLdouble radius,
                     GLint slices, GLint stacks);
void glutWireSphere(GLdouble radius,
                    GLint slices, GLint stacks);
\end{verbatim}
\begin{description}
\itemsep 0in
\item[{\tt radius}]
The radius of the sphere.
\item[{\tt slices}]
The number of subdivisions around the Z axis (similar to lines of longitude).
\item[{\tt stacks}]
The number of subdivisions along the Z axis (similar to lines of latitude).
\end{description}

\subsubsection*{Description}

Renders a sphere centered at the modeling coordinates origin of the specified {\tt radius}.
The sphere is subdivided around the Z axis into slices and along the Z
axis into stacks.

\subsection{glutSolidCube, glutWireCube}

\rndx{glutSolidCube} and \rndx{glutWireCube} render a solid or wireframe
cube respectively.

\subsubsection*{Usage}
\begin{verbatim}
void glutSolidCube(GLdouble size);
void glutWireCube(GLdouble size);
\end{verbatim}
\begin{description}
\itemsep 0in
\item[{\tt size}]
Length of each edge.
\end{description}

\subsubsection*{Description}

{\tt glutSolidCube} and {\tt glutWireCube} render a solid or wireframe
cube respectively.
The cube is centered at the modeling coordinates origin with sides of length {\tt size}.

\subsection{glutSolidCone, glutWireCone}

\rndx{glutSolidCone} and \rndx{glutWireCone} render a solid or wireframe
cone respectively.

\subsubsection*{Usage}
\begin{verbatim}
void glutSolidCone(GLdouble base, GLdouble height,
                   GLint slices, GLint stacks);
void glutWireCone(GLdouble base, GLdouble height,
                  GLint slices, GLint stacks);
\end{verbatim}
\begin{description}
\itemsep 0in
\item[{\tt base}]
The radius of the base of the cone.
\item[{\tt height}]
The height of the cone.
\item[{\tt slices}]
The number of subdivisions around the Z axis.
\item[{\tt stacks}]
The number of subdivisions along the Z axis.
\end{description}

\subsubsection*{Description}

{\tt glutSolidCone} and {\tt glutWireCone} render a solid or wireframe
cone respectively oriented along the Z axis.  The base of the cone
is placed at Z = 0, and the top at Z = {\tt height}.  The cone is subdivided
around the Z axis into slices, and along the Z axis into stacks.

\subsection{glutSolidTorus, glutWireTorus}

\rndx{glutSolidTorus} and \rndx{glutWireTorus} render a solid or wireframe
torus (doughnut) respectively.

\subsubsection*{Usage}
\begin{verbatim}
void glutSolidTorus(GLdouble innerRadius,
                    GLdouble outerRadius,
                    GLint nsides, GLint rings);
void glutWireTorus(GLdouble innerRadius,
                   GLdouble outerRadius,
                   GLint nsides, GLint rings);
\end{verbatim}
\begin{description}
\itemsep 0in
\item[{\tt innerRadius}]
Inner radius of the torus.
\item[{\tt outerRadius}]
Outer radius of the torus.
\item[{\tt nsides}]
Number of sides for each radial section.
\item[{\tt rings}]
Number of radial divisions for the torus.
\end{description}

\subsubsection*{Description}

{\tt glutSolidTorus} and {\tt glutWireTorus} render a solid or wireframe
torus (doughnut) respectively centered at the modeling coordinates origin whose axis is aligned
with the Z axis.

\subsection{glutSolidDodecahedron, glutWireDodecahedron}

\rndx{glutSolidDodecahedron} and \rndx{glutWireDodecahedron} render a solid or wireframe
dodecahedron (12-sided regular solid) respectively.

\subsubsection*{Usage}
\begin{verbatim}
void glutSolidDodecahedron(void);
void glutWireDodecahedron(void);
\end{verbatim}

\subsubsection*{Description}

{\tt glutSolidDodecahedron} and {\tt glutWireDodecahedron} render a solid or wireframe
dodecahedron respectively centered at the modeling coordinates origin with a radius of $\sqrt{3}$.

\subsection{glutSolidOctahedron, glutWireOctahedron}

\rndx{glutSolidOctahedron} and \rndx{glutWireOctahedron} render a solid or
wireframe octahedron (8-sided regular solid) respectively.

\subsubsection*{Usage}
\begin{verbatim}
void glutSolidOctahedron(void);
void glutWireOctahedron(void);
\end{verbatim}

\subsubsection*{Description}

{\tt glutSolidOctahedron} and {\tt glutWireOctahedron} render a solid or
wireframe octahedron respectively centered at the modeling coordinates origin with a radius of 1.0.

\subsection{glutSolidTetrahedron, glutWireTetrahedron}

\rndx{glutSolidTetrahedron} and \rndx{glutWireTetrahedron} render a solid or wireframe
tetrahedron (4-sided regular solid) respectively.

\subsubsection*{Usage}
\begin{verbatim}
void glutSolidTetrahedron(void);
void glutWireTetrahedron(void);
\end{verbatim}

\subsubsection*{Description}

{\tt glutSolidTetrahedron} and {\tt glutWireTetrahedron} render a solid or wireframe
tetrahedron respectively centered at the modeling coordinates origin with a radius of $\sqrt{3}$.

\subsection{glutSolidIcosahedron, glutWireIcosahedron}

\rndx{glutSolidIcosahedron} and \rndx{glutWireIcosahedron} render a solid or wireframe
icosahedron (20-sided regular solid) respectively.

\subsubsection*{Usage}
\begin{verbatim}
void glutSolidIcosahedron(void);
void glutWireIcosahedron(void);
\end{verbatim}

\subsubsection*{Description}

{\tt glutSolidIcosahedron} and {\tt glutWireIcosahedron} render a solid or wireframe
icosahedron respectively.  The icosahedron is centered at the modeling coordinates origin and has a radius
of 1.0.

\subsection{glutSolidTeapot, glutWireTeapot}

\rndx{glutSolidTeapot} and \rndx{glutWireTeapot} render a solid or wireframe
teapot\footnote{Yes, the {\em classic} computer graphics teapot modeled
by Martin Newell in 1975 \cite{crow87}.} respectively.

\subsubsection*{Usage}
\begin{verbatim}
void glutSolidTeapot(GLdouble size);
void glutWireTeapot(GLdouble size);
\end{verbatim}
\begin{description}
\itemsep 0in
\item[{\tt size}] Relative size of the teapot.
\end{description}

\subsubsection*{Description}

{\tt glutSolidTeapot} and {\tt glutWireTeapot} render a solid or
wireframe teapot respectively.  Both surface normals and texture
coordinates for the teapot are generated.  The teapot is generated with
OpenGL evaluators.

